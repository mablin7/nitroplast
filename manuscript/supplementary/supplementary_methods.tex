\documentclass[11pt, a4paper]{article}

% ============================================================================
% PACKAGES
% ============================================================================

\usepackage[utf8]{inputenc}
\usepackage[T1]{fontenc}
\usepackage{microtype}
\usepackage[margin=2.5cm]{geometry}
\usepackage{amsmath, amssymb}
\usepackage{booktabs}
\usepackage{longtable}
\usepackage[colorlinks=true, linkcolor=blue!60!black, citecolor=blue!60!black, urlcolor=blue!60!black]{hyperref}
\usepackage[round, sort&compress]{natbib}
\usepackage[font=small, labelfont=bf]{caption}

% Section styling
\usepackage{titlesec}
\titleformat{\section}{\large\bfseries}{S\thesection}{1em}{}
\titleformat{\subsection}{\normalsize\bfseries}{S\thesubsection}{1em}{}

% Paragraph spacing
\setlength{\parskip}{0.5em}
\setlength{\parindent}{0pt}

% ============================================================================
% DOCUMENT
% ============================================================================

\begin{document}

\begin{center}
{\Large\bfseries Supplementary Methods}\\[1em]
{\large The UCYN-A transit peptide: a novel C-terminal targeting system\\ with distinctive biophysical substrate signatures}
\end{center}

\vspace{2em}

\tableofcontents

\newpage

% ============================================================================
\section{Computational Environment}
% ============================================================================

All analyses were performed using Python 3.12. Package management used uv (version 0.5.x). Key dependencies and versions:

\begin{longtable}{ll}
\toprule
\textbf{Package} & \textbf{Version} \\
\midrule
\endhead
BioPython & 1.84 \\
NumPy & 2.0.x \\
pandas & 2.2.x \\
scikit-learn & 1.5.x \\
SciPy & 1.14.x \\
matplotlib & 3.9.x \\
seaborn & 0.13.x \\
PyTorch & 2.4.x \\
transformers & 4.44.x \\
h5py & 3.11.x \\
statsmodels & 0.14.x \\
\bottomrule
\end{longtable}

% ============================================================================
\section{Data Sources and Preprocessing}
% ============================================================================

\subsection{uTP protein identification}

We used the uTP HMM profile from Coale et al. to scan the \textit{B.~bigelowii} transcriptome using HMMER 3.4. Parameters:

\begin{itemize}
\item E-value threshold: 0.01
\item Minimum bit score: 30.0
\item HMM coverage requirement: match must start within first 50 positions of the profile
\end{itemize}

This identified 933 import candidates. The HMM hit position was used to define the boundary between mature domain and uTP region for each protein.

\subsection{Sequence datasets}

\begin{longtable}{lrl}
\toprule
\textbf{Dataset} & \textbf{Count} & \textbf{Description} \\
\midrule
\endhead
Import candidates & 933 & HMM-predicted uTP proteins \\
Experimentally validated & 206 & Enriched in UCYN-A proteomics \\
\textit{B.~bigelowii} proteome & 44,363 & Full predicted proteome \\
\bottomrule
\end{longtable}

% ============================================================================
\section{uTP Sequence Organization}
% ============================================================================

\subsection{Motif discovery}

Motif discovery was performed using MEME Suite 5.5.5 on 206 experimentally validated uTP sequences. The C-terminal regions were extracted and filtered using Gblocks to remove poorly aligned positions. MEME parameters:

\begin{itemize}
\item Mode: zoops (zero or one occurrence per sequence)
\item Number of motifs: 10
\item Minimum motif width: 6
\item Maximum motif width: 50
\item Background model: order-0 Markov from input sequences
\end{itemize}

\subsection{Motif scanning}

We scanned all 933 import candidates for the discovered motifs using MAST (Motif Alignment and Search Tool). Parameters:

\begin{itemize}
\item E-value threshold: 10.0 (permissive, to detect weak hits)
\item Output: hit positions, p-values, and motif order
\end{itemize}

Motif patterns were classified by terminal motif identity:
\begin{itemize}
\item terminal\_4: sequences ending with motif 4
\item terminal\_5: sequences ending with motif 5
\item terminal\_7: sequences ending with motif 7
\item terminal\_9: sequences ending with motif 9
\item other: sequences with non-standard terminal motifs
\end{itemize}

% ============================================================================
\section{Structure Prediction and Analysis}
% ============================================================================

\subsection{AlphaFold3 structure prediction}

Structures were predicted using AlphaFold3 via the AlphaFold Server (\url{https://alphafoldserver.com}). We selected 47 proteins with high-confidence uTP regions (good C-terminal alignment in the multiple sequence alignment).

\subsection{Structural alignment}

Structures were aligned using the CE (Combinatorial Extension) algorithm implemented in PyMOL. The C-terminal uTP regions (approximately 120 residues) were extracted and aligned to the structure with the longest C-terminal chain as reference.

Pairwise RMSD was computed using BioPython's Superimposer class on C$\alpha$ atoms.

\subsection{Consensus structure}

The consensus structure was built as follows:

\begin{enumerate}
\item Identify reference structure (longest chain)
\item For each reference residue position, find spatially corresponding residues across all structures (within 3.0~\AA{} of the reference position)
\item Compute consensus position as the mean of all corresponding C$\alpha$ coordinates
\item Record positional standard deviation as a measure of variance
\end{enumerate}

\subsection{Secondary structure assignment}

Secondary structure was assigned using DSSP algorithm criteria applied to backbone dihedral angles. Helices were defined as regions with $\geq$4 consecutive residues in helical conformation ($\phi \approx -60°$, $\psi \approx -45°$).

% ============================================================================
\section{Sequence Space Analysis}
% ============================================================================

\subsection{Sequence embeddings}

Protein sequences were encoded using the ProtT5-XL-UniRef50 model. For each sequence:

\begin{enumerate}
\item Tokenize sequence using the ProtT5 tokenizer
\item Pass through the encoder (no decoder)
\item Extract per-residue embeddings from the final hidden layer
\item Mean-pool across residue positions to obtain a single 1024-dimensional vector
\end{enumerate}

Embeddings were computed on GPU (NVIDIA RTX PRO 6000) and cached in HDF5 format.

\subsection{Clustering methods}

Four clustering methods were applied:

\begin{longtable}{lll}
\toprule
\textbf{Method} & \textbf{Implementation} & \textbf{Parameters} \\
\midrule
\endhead
Hierarchical & scipy.cluster.hierarchy & k=4,5,6,7; linkage=average \\
Spectral & sklearn.cluster.SpectralClustering & k=4,5,6,7; affinity=rbf \\
K-means & sklearn.cluster.KMeans & k=4,5,6,7; n\_init=10 \\
DBSCAN & sklearn.cluster.DBSCAN & eps=0.5,1.0,2.0; min\_samples=5 \\
\bottomrule
\end{longtable}

For hierarchical clustering, pairwise distances were computed as Jaccard distance on k-mer frequency vectors (k=4).

\subsection{Clustering evaluation}

Cluster quality was assessed using:

\begin{itemize}
\item \textbf{Silhouette score}: Mean silhouette coefficient across all samples. Range $[-1, +1]$; values $>0.5$ indicate strong cluster structure.
\item \textbf{Adjusted Rand Index (ARI)}: Agreement between two clusterings, adjusted for chance. Range $[-1, +1]$; values near 0 indicate random agreement.
\end{itemize}

\subsection{Null model comparison}

To test whether uTP sequences show more structure than expected by chance:

\begin{enumerate}
\item For each real uTP sequence, generate a null sequence by randomly shuffling amino acids (preserving exact composition)
\item Compute ProtT5 embeddings for all null sequences
\item Apply k-means clustering (k=4) and compute silhouette score
\item Repeat for 100 independent null sequence sets
\item Compare observed silhouette score to null distribution using one-sided permutation test
\end{enumerate}

Additional metrics computed: Hopkins statistic, mean pairwise distance, distance variance.

% ============================================================================
\section{Mature Domain Classification}
% ============================================================================

\subsection{Mature domain extraction}

For each uTP protein, the mature domain was defined as the sequence from the N-terminus to the start of the HMM hit position. Proteins with mature domains $<$30 or $>$3000 amino acids were excluded.

\subsection{Control group selection}

Control proteins were selected from the \textit{B.~bigelowii} proteome:

\begin{enumerate}
\item Exclude all proteins with HMM hits (uTP candidates)
\item Select candidates with lengths matching the mature domain length distribution
\item Submit to CELLO 2.5 (\url{http://cello.life.nctu.edu.tw/}) for subcellular localization prediction
\item Organism type: Eukaryote
\item Retain only proteins predicted as cytoplasmic or nuclear
\end{enumerate}

This yielded 773 control proteins.

\subsection{Feature extraction}

Features were extracted using ProtT5-XL-UniRef50 as described above, yielding 1024-dimensional embeddings for each mature domain.

\subsection{Classifier training}

\begin{longtable}{ll}
\toprule
\textbf{Parameter} & \textbf{Value} \\
\midrule
\endhead
Algorithm & Logistic Regression \\
Regularization & L2 (C=1.0) \\
Class weights & Balanced \\
Cross-validation & 5-fold stratified \\
Scaling & StandardScaler (mean=0, std=1) \\
\bottomrule
\end{longtable}

\subsection{Statistical validation}

\textbf{Permutation test}: Labels were randomly shuffled 1000 times. For each permutation, a classifier was trained and evaluated using 5-fold CV. The p-value is the fraction of permuted accuracies $\geq$ observed accuracy.

\textbf{Bootstrap confidence intervals}: 1000 bootstrap samples were drawn with replacement. For each sample, the classifier was trained and evaluated, yielding a distribution of accuracy estimates. 95\% CI computed as the 2.5th and 97.5th percentiles.

\textbf{Full proteome validation}: The classifier trained on the balanced dataset was applied to predict uTP status for all 44,363 \textit{B.~bigelowii} proteins. ROC AUC was computed using true labels (933 uTP, 43,430 non-uTP).

% ============================================================================
\section{Biophysical Property Analysis}
% ============================================================================

\subsection{Property calculation}

Properties were computed using BioPython's ProteinAnalysis class:

\begin{longtable}{lll}
\toprule
\textbf{Property} & \textbf{Method} & \textbf{Description} \\
\midrule
\endhead
Isoelectric point & isoelectric\_point() & pH at which net charge = 0 \\
Instability index & instability\_index() & Guruprasad et al. formula \\
GRAVY & gravy() & Grand average of hydropathy \\
Molecular weight & molecular\_weight() & Sum of residue masses \\
\bottomrule
\end{longtable}

\subsection{Disorder prediction}

Intrinsic disorder was estimated using secondary structure propensity. Fraction coil was computed as the proportion of residues with low helix and sheet propensity based on the Chou-Fasman parameters.

\subsection{Effect size calculation}

Cohen's d was computed as:
\[
d = \frac{\bar{x}_1 - \bar{x}_2}{s_{\text{pooled}}}
\]

where $s_{\text{pooled}} = \sqrt{\frac{(n_1-1)s_1^2 + (n_2-1)s_2^2}{n_1+n_2-2}}$

Interpretation: $|d| = 0.2$ (small), $|d| = 0.5$ (medium), $|d| = 0.8$ (large).

\subsection{Multiple testing correction}

Bonferroni correction was applied for 8 pre-specified property comparisons ($\alpha = 0.05/8 = 0.00625$).

% ============================================================================
\section{Gene Family Analysis}
% ============================================================================

\subsection{Sequence representation}

Each protein was represented as a k-mer frequency vector:

\begin{itemize}
\item k-mer size: 3
\item Vector dimension: $20^3 = 8000$
\item Normalization: L1 (frequencies sum to 1)
\end{itemize}

\subsection{Distance calculation}

Pairwise Jaccard distance was computed:
\[
d(A, B) = 1 - \frac{|A \cap B|}{|A \cup B|}
\]

where $A$ and $B$ are the sets of k-mers present in each sequence (frequency $>$ 0).

\subsection{Hierarchical clustering}

Clustering was performed using scipy.cluster.hierarchy with:

\begin{itemize}
\item Method: average linkage
\item Distance threshold: 0.7 (corresponding approximately to 40\% sequence identity)
\end{itemize}

\subsection{Permutation test}

To test whether uTP proteins cluster more than expected by chance:

\begin{enumerate}
\item Compute observed metrics:
\begin{itemize}
\item Fraction of uTP proteins sharing a family with another uTP protein
\item Number of distinct families containing uTP proteins
\item Maximum uTP proteins in any single family
\end{itemize}
\item Randomly reassign uTP labels among all proteins (preserving total count)
\item Recompute metrics
\item Repeat 10,000 times
\item P-value = fraction of permuted values $\geq$ observed (for clustering metrics) or $\leq$ observed (for diversity metrics)
\end{enumerate}

% ============================================================================
\section{Functional Enrichment and Within-Category Analysis}
% ============================================================================

\subsection{Functional annotation}

Proteins were annotated using the existing \textit{B.~bigelowii} transcriptome annotations from Coale et al., which include COG (Clusters of Orthologous Groups) categories assigned via eggNOG-mapper.

\subsection{Within-category comparison}

For each COG category with $\geq$10 uTP proteins and $\geq$10 control proteins:

\begin{enumerate}
\item Compute Cohen's d for each biophysical property within the category
\item Compare to overall effect size (pooling across categories)
\item Compute percent of effect explained by function:
\[
\text{\% explained} = \frac{d_{\text{overall}} - d_{\text{within}}}{d_{\text{overall}}} \times 100
\]
\end{enumerate}

Eight categories met the sample size criterion.

\subsection{Variance partitioning}

Variance partitioning was performed using Type II ANOVA:

\[
\text{Property} \sim \text{uTP\_status} + \text{COG\_category}
\]

Variance components:
\begin{itemize}
\item \textbf{uTP unique}: $\text{SS}_{\text{uTP}} / \text{SS}_{\text{total}}$
\item \textbf{Function unique}: $\text{SS}_{\text{COG}} / \text{SS}_{\text{total}}$
\item \textbf{Shared}: Computed via sequential decomposition
\item \textbf{Unexplained}: $\text{SS}_{\text{residual}} / \text{SS}_{\text{total}}$
\end{itemize}

\subsection{Meta-analysis of within-category effects}

Heterogeneity across categories was assessed using the $I^2$ statistic:
\[
I^2 = \frac{Q - (k-1)}{Q} \times 100\%
\]

where $Q$ is Cochran's Q statistic and $k$ is the number of categories.

Interpretation: $I^2 < 25\%$ (low), $25\% \leq I^2 < 75\%$ (moderate), $I^2 \geq 75\%$ (high heterogeneity).

% ============================================================================
\section{Code Availability}
% ============================================================================

All analysis scripts are available at [repository URL]. The repository includes:

\begin{itemize}
\item \texttt{experiments/utp\_motif\_analysis/} -- Motif discovery and coverage
\item \texttt{experiments/utp\_consensus\_structure/} -- Structure prediction and alignment
\item \texttt{experiments/utp\_sequence\_clustering/} -- Sequence space analysis
\item \texttt{experiments/utp\_structure\_vs\_null/} -- Null model comparison
\item \texttt{experiments/utp\_presence\_classifier/} -- Mature domain classifier
\item \texttt{experiments/utp\_family\_clustering/} -- Gene family analysis
\item \texttt{experiments/utp\_functional\_annotation/} -- Functional enrichment
\end{itemize}

Each experiment directory contains a README.md with execution instructions and expected outputs.

% ============================================================================
% REFERENCES (if needed)
% ============================================================================

% \bibliographystyle{abbrvnat}
% \bibliography{../references}

\end{document}
