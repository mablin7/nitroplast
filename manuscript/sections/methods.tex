\section{Methods}

% ============================================================================
\subsection{Data sources}
% ============================================================================

We obtained proteomics data, genome annotations, and the uTP hidden Markov model (HMM) profile from Coale et al.~\citep{coaleNitrogenfixingOrganelleMarine2024}. The \textit{B.~bigelowii} transcriptome contains approximately 44,000 predicted proteins. Scanning the full proteome with the uTP HMM identified 933 proteins with significant hits (e-value $<$ 0.01), which we term import candidates. Of these, 206 were experimentally validated as enriched inside UCYN-A by quantitative proteomics.

% ============================================================================
\subsection{uTP sequence organization}
% ============================================================================

We used motif discovery to identify conserved elements within uTP sequences. Starting from the 206 experimentally validated proteins, we identified ten sequence motifs within the C-terminal extension. Two motifs (termed anchor~1 and anchor~2) appear near the start of the uTP region and are present in the majority of sequences. We extended this analysis to all 933 import candidates by scanning for these motifs, detecting hits in 745 proteins (80\%). Among proteins with detectable motifs, 60\% display anchor~2 preceding anchor~1, which we define as the canonical order.

% ============================================================================
\subsection{Structure prediction and analysis}
% ============================================================================

We predicted three-dimensional structures for 47 uTP-containing proteins using AlphaFold3. Structures were aligned using structural superposition, and we computed pairwise root mean square deviation (RMSD) across all pairs. To quantify conservation, we built a consensus structure by averaging atomic positions across aligned structures and calculated positional variance (standard deviation) at each residue position.

Secondary structure assignments were made for each predicted structure. We mapped the anchor motifs to structural elements by comparing motif positions in sequence to helix boundaries in the predicted structures.

% ============================================================================
\subsection{Sequence space analysis}
% ============================================================================

We tested whether uTP sequences form discrete clusters using four clustering methods: hierarchical clustering with k-mer distance, spectral clustering, k-means on protein language model embeddings, and density-based clustering. We evaluated cluster quality using silhouette scores (range $-$1 to $+$1, with values above 0.5 indicating strong cluster structure) and compared cluster assignments across methods using the adjusted Rand index.

To test whether uTP sequences show more or less structure than expected by chance, we compared real sequences to null sequences generated by shuffling each sequence while preserving amino acid composition. For each of 100 null sequence sets, we computed the silhouette score and compared the distribution to the observed value using a permutation test.

% ============================================================================
\subsection{Mature domain classifier}
% ============================================================================

We trained a classifier to predict uTP presence from mature domain sequences (the functional protein excluding the C-terminal uTP region). The uTP region was identified using the HMM profile and removed, yielding 605 mature domain sequences.

For the control group, we selected proteins from the \textit{B.~bigelowii} proteome that lack uTP. To avoid confounding by proteins with other targeting signals, we filtered candidates using subcellular localization prediction, retaining only proteins predicted to localize to the cytoplasm or nucleus (773 proteins). This ensures that differences between groups reflect uTP-specific properties rather than general features of targeted proteins.

Features were extracted using a protein language model, which encodes each sequence as a 1024-dimensional vector capturing evolutionary and structural information. We trained a logistic regression classifier using five-fold cross-validation with stratified sampling to maintain class proportions. Classifier significance was assessed by permutation testing (1000 permutations). We validated the classifier on the full proteome (933 uTP versus 43,430 non-uTP proteins) and report the area under the receiver operating characteristic curve.

% ============================================================================
\subsection{Biophysical property analysis}
% ============================================================================

We computed biophysical properties for all mature domains: isoelectric point, instability index, and fraction of residues in disordered regions (predicted coil). Effect sizes were quantified using Cohen's d, with values of 0.2, 0.5, and 0.8 corresponding to small, medium, and large effects. All comparisons were corrected for multiple testing using Bonferroni correction.

% ============================================================================
\subsection{Gene family analysis}
% ============================================================================

To assess whether uTP proteins share common ancestry, we clustered all \textit{B.~bigelowii} proteins into gene families based on mature domain sequence similarity. We used k-mer frequency vectors (k=3) and hierarchical clustering with a distance threshold corresponding approximately to 40\% sequence identity. We then asked whether uTP proteins are more likely to share gene families than expected by chance. The null expectation was estimated by permutation testing (10,000 permutations), randomly reassigning uTP labels while preserving the total number of uTP proteins.

% ============================================================================
\subsection{Functional enrichment and within-category analysis}
% ============================================================================

We assigned proteins to functional categories using COG (Clusters of Orthologous Groups) annotations. To test whether biophysical properties are confounded by function, we compared uTP versus control proteins within each functional category that contained at least ten proteins from each group (eight categories met this criterion). We computed effect sizes within each category and compared them to overall effect sizes. If the biophysical signature were explained by functional enrichment, within-category effect sizes should approach zero.

We performed variance partitioning to quantify how much of the biophysical variation is explained by uTP status, functional category, and their overlap. This analysis decomposes total variance into unique contributions from each factor plus shared variance.

% ============================================================================
\subsection{Statistical framework}
% ============================================================================

Throughout this study, we report both p-values and effect sizes. For continuous comparisons, we use Cohen's d; for classifier performance, we report accuracy, area under the ROC curve, and 95\% confidence intervals from bootstrap resampling. Multiple testing correction uses Bonferroni (for pre-specified comparisons) or Benjamini-Hochberg false discovery rate (for exploratory analyses). Permutation tests use 1000 iterations unless otherwise specified. Detailed methods including software versions and parameters are provided in Supplementary Methods.
