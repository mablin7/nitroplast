\section{Introduction}

Mitochondria and chloroplasts import the vast majority of their proteins from the host cell cytoplasm. This import is mediated by N-terminal transit peptides that are recognized by dedicated translocon complexes (TOM/TIM in mitochondria, TOC/TIC in chloroplasts) and typically cleaved upon import \citep{TODO}. These targeting systems are ancient, highly conserved, and essential for organelle function. The evolution of efficient protein import was a key innovation that enabled the extensive genome reduction characteristic of endosymbiont-to-organelle transitions.

UCYN-A (\textit{Candidatus} Atelocyanobacterium thalassa) is a nitrogen-fixing cyanobacterial endosymbiont that has recently been characterized as an organelle in the marine haptophyte \textit{Braarudosphaera bigelowii} \citep{coale2024}. The symbiosis originated approximately 90--100 million years ago, and UCYN-A has undergone extreme genome reduction, retaining only $\sim$22\% of genes compared to its free-living relative \textit{Crocosphaera watsonii} \citep{frail2025}. This makes UCYN-A—termed the ``nitroplast''—the first nitrogen-fixing organelle to be characterized in eukaryotes.

Coale et al. identified approximately 368 host-encoded proteins that are imported into UCYN-A, representing a substantial complement of the nitroplast proteome \citep{coale2024}. These imported proteins carry a novel C-terminal targeting sequence of $\sim$100--150 amino acids, termed the UCYN-A transit peptide (uTP). Using a hidden Markov model (HMM) derived from experimentally validated proteins, approximately 900 proteins in the \textit{B. bigelowii} proteome are predicted to carry uTP. The imported proteins fill critical metabolic gaps in UCYN-A, including biosynthesis of threonine, serine, proline, pyrimidines, and tetrahydrofolate.

Despite the identification of uTP as a targeting signal, several fundamental questions remain unanswered. What are the conserved sequence and structural features of uTP? Why is uTP C-terminal, in contrast to the N-terminal signals used by mitochondria and chloroplasts? What is the evolutionary origin of uTP—was it co-opted from an existing cellular function or did it evolve de novo? And what determines which host proteins acquire uTP for nitroplast targeting?

Here, we present a systematic computational characterization of the uTP system. We identify conserved motif architecture with invariant anchor sequences and a conserved U-bend structure. We demonstrate that uTP has no detectable homologs in related haptophyte genomes, supporting a de novo evolutionary origin. Most importantly, we show that uTP-containing proteins share a distinctive biophysical signature—more disordered, acidic, and stable than the general proteome—that enables prediction of uTP status with 90\% accuracy from mature domain sequence alone. These findings establish uTP as a novel solution to organellar protein targeting and provide a foundation for understanding the mechanisms underlying nitroplast protein import.
