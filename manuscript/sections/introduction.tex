\section{Introduction}

Nitrogen limits the growth of most terrestrial and aquatic ecosystems. Although molecular nitrogen constitutes 78\% of the atmosphere, only prokaryotes possess the enzymatic machinery to reduce it to biologically available ammonia. This metabolic capability, nitrogen fixation, has never evolved in any eukaryote~\cite{piOriginEvolutionNitrogen}. Engineering nitrogen-fixing crops remains a long-standing goal that would reduce dependence on industrial fertilizer production and its associated environmental costs. Yet despite decades of effort, the barriers to transferring nitrogen fixation into eukaryotic cells remain formidable.

Nature has solved this problem exactly once. In 2024, Coale and colleagues demonstrated that the cyanobacterial endosymbiont UCYN-A (\textit{Candidatus} Atelocyanobacterium thalassa) has crossed the threshold from endosymbiont to organelle in its marine haptophyte host \textit{Braarudosphaera bigelowii}~\cite{coaleNitrogenfixingOrganelleMarine2024}. They named this nitrogen-fixing organelle the nitroplast. Three lines of evidence support this classification: UCYN-A divides in synchrony with its host cell~\cite{turk-kuboSoftXrayTomography2023}, UCYN-A has undergone extreme genome reduction that renders it metabolically dependent on the host~\cite{masudaCrocosphaeraWatsoniiWidespread2024}, and critically, the host imports hundreds of nuclear-encoded proteins into UCYN-A. The nitroplast represents the first opportunity to study organellogenesis as it unfolds, rather than inferring the process from the highly derived mitochondria and chloroplasts that emerged billions of years ago.

The protein import system is central to both understanding organellogenesis and any future engineering efforts. When an endosymbiont becomes an organelle, genes transfer from endosymbiont to host nucleus while their protein products must still reach the organelle to function~\cite{frailGenomesNitrogenfixingEukaryotes2025}. This requirement creates intense selective pressure for targeting mechanisms. Mitochondria and chloroplasts solved this problem through N-terminal transit peptides recognized by elaborate translocon complexes. The nitroplast evolved an independent solution: a C-terminal extension of approximately 120 amino acids that Coale et al. termed the UCYN-A transit peptide (uTP)~\cite{coaleNitrogenfixingOrganelleMarine2024}. This targeting sequence has no detectable sequence or structural homologs in any public database, suggesting it arose de novo during the nitroplast endosymbiosis.

The discovery of the uTP system raises fundamental questions about how novel protein import mechanisms originate. What structural features does the import machinery recognize? Did uTP spread through the host proteome in discrete acquisition events, or did it expand gradually? And crucially for any engineering application: can any protein be targeted for import by adding uTP, or do cargo proteins themselves require specific properties? Comparative analysis with the related diazoplast system in \textit{Epithemia} diatoms, where minimal protein import has evolved despite millions of years of endosymbiosis~\cite{frailGenomesNitrogenfixingEukaryotes2025}, suggests that successful import systems require more than simply acquiring a targeting signal.

Here we characterize the architecture and evolutionary dynamics of the uTP system. We find that uTP sequences combine conserved structural elements with continuous sequence variation, arguing against discrete acquisition events. Unexpectedly, we discover that biophysical properties of the mature protein domain predict uTP presence with high accuracy. This finding suggests dual constraints on uTP-mediated import: the transit peptide must present conserved structural features for recognition, while the cargo protein must possess compatible biophysical properties. These constraints illuminate both the evolutionary trajectory of the nitroplast and the challenges facing efforts to engineer similar systems.

